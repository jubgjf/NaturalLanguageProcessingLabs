\section{相关工作}

基础的分词算法可分为:
基于字符串匹配\citep{张启宇2008中文分词算法研究综述}、
基于统计的分词方法\citep{刘件2008中文分词算法研究}和
基于理解的分词方法\citep{汪文妃2018中文分词算法研究综述};

基于字符串匹配的分词方法:又叫做机械分词方法。它是将待分析的字符串与一个充分大的词典进行匹配,
若在词典中找到某个字符串,则匹配成功(识别出一个词):目前存在正向最大匹配法、逆向最大匹配法、
最少切分和双向最大匹配法分词方法。

基于统计的分词方法:给出大量已经分词的文本,利用统计机器学习模型学习词语切分的规律
,从而实现对未知文本的切分。主要统计模型有:N元文法模型,隐马尔可夫模型,最大熵模型,
条件随机场模型等。

实验实现的功能如表\ref{achieved-features}。

\begin{table}[H]
  \centering
  \begin{tabular}{lr}
    \hline
    \textbf{功能}                & \textbf{指导书章节} \\
    \hline
    词典的构建                   & 3.1                 \\
    正反向最大匹配分词的实现     & 3.2                 \\
    正反向最大匹配分词的效果分析 & 3.3                 \\
    机械匹配的分词系统的速度优化 & 3.4                 \\
    基于统计的一元文法分词       & 3.5                 \\
    基于统计的二元文法分词       & 3.5                 \\
    二元文法未登录词识别         & 3.6                 \\
    \hline
  \end{tabular}
  \caption{实验实现的功能}
  \label{achieved-features}
\end{table}
